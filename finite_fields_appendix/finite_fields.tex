\graphicspath{{finite_fields/}}


\chapter{Finite Fields}
\label{appendix:finite_fields}

In this appendix, we briefly review the theory of finite fields\footnote{We assume the reader has 
a very basic knowledge of some algebraic objects like Groups, Rings, and Ideals.}. A field $K$ is 
a set equipped with two operations $+ : K \times K \rightarrow K$, and $\times : K \times K 
\rightarrow K$ called addition and multiplication. The following conditions are imposed 
for all $a, b, c \in K$:
\begin{enumerate}
	\setlength\itemsep{0mm}
	\item Associativity: $a \times (b \times c) = (a \times b) \times c, a + (b + c) = (a + b) + c$
	\item Commutativity: $a \times b = b \times a, a + b = b + a$
	\item Identity: there exist elements denoted by $0, 1$ in $K$ such that $a \times 1 = a$, and 
	$a + 0 = a$.
	\item Inverse: for $a \ne 0$, there exist $-a, a^{-1} \in K$ such that $a + (-a) = 0$, and $a 
	\times a^{-1} =	1$.
	\item Distributivity: $a \times (b + c) = a \times b + a \times c$.
\end{enumerate}
We usually use familiar notations for the above two operations, e.g. the multiplication symbol is
often skipped. It is apparent from the above conditions that the elements $K^* = K \backslash 0$ 
form a multiplicative group. The multiplicative order of an element $a \in K$ is a the smallest 
positive integer $n$ (if exists) such that $a^n = 1$. The characteristic of a field $K$, denoted 
by $\fieldchar(K)$, is the smallest positive integer $n$ (if exists) such that
\[ \underbrace{1 + \cdots + 1}_{n \text{ times}} = 0. \]
If such integer does not exists, the characteristic is $\infty$. For any field $K$, 
$\fieldchar(K)$ is either $\infty$ or a prime number. A homomorphism $\varphi: E \rightarrow F$ of 
fields is a homomorphism of $E, F$ considered as rings. It preserves addition and multiplication. 
More precisely, $\varphi(ab) = \varphi(a)\varphi(b)$, and $\varphi(a + b) = \varphi(a) + 
\varphi(b)$ for $a, b \in E$.

\paragraph{Polynomial ring.} Given a ring $R$, the set of univariate polynomial with coefficients 
in $R$ is a ring denoted by $R[X]$. The ring of bivariate polynomials is defined by $R[X, Y] = 
R[X][Y]$, and polynomial rings with higher number of variables are defined inductively. When $R$ is 
a field $K$ then every ideal of $K[X]$ is of the form $\ang{f}$, i.e. it is generated by a 
polynomial $f \in K[X]$. A polynomial $f \in K[X]$ is irreducible if it cannot be written as $f = 
gh$ with $\deg g, \deg h > 0$. The ideal $\ang{f}$ is prime if and only if the polynomial $f$ is 
irreducible. In that case the quotient $K[X] / \ang{f}$ is a field. An element $a$ in some 
extension $F$ of $K$ is a root of $f \in K[X]$ if $(X - a)$ is a factor of $f$ in $F[X]$ or 
equivalently if $f(a) = 0$. Each term of a polynomial is called a monomial. The monomial of the 
highest degree is called the leading term, and it coefficient is called the leading coefficient. A 
monic polynomial is a polynomial with leading coefficient equal to 1. Over a field every 
polynomial can be made monic by multiplying it by the inverse of its leading coefficient.
\vspace*{3mm}

A \textbf{\textit{finite field}} is a field with a finite number of elements. The most familiar 
finite fields are the prime fields. Given a prime number $p$, a prime finite field, denoted by 
$\F_p$, is a field consisting of numbers $\{0, 1, \dots, p - 1\}$. The operations are done module 
$p$, and $p$ is called the modulus. Any field $K$ with $\fieldchar(K) = p$ contains a copy of 
$\F_p$. In other words, $K$ is a vector space over $\F_p$. This means the cardinality of a finite 
field is always a prime power, namely $p^{[K: \F_p]}$, where $[K: \F_p]$ is the degree of the 
extension $\F_p \subseteq K$.


\section{Basic properties}

Assume a finite field $F$ has a subfield $E \subseteq F$ of size $q$. If the degree of the 
extension $E \subseteq F$ is $n$ then $F$ has $q^n$ elements. Indeed, the elements of $F$ can be 
represented as unique sums $a_1x_1 + a_2x_2 + \cdots + a_nx_n$ where $a_i \in E$ and $\{x_i\}_i$ 
is a basis of $E$ over $F$ as a vector space. We shall denote a finite field of size $q$, where 
$q = p^n$ is a prime power, by $\F_q$. Every element $a \in \F_q$ satisfies $a^q = a$, since the 
multiplicative group $\F_q^*$ has size $q - 1$. In other words, every element of $\F_q$ is a root 
of the polynomial $g(X) = X^q - X \in \F_p[X]$. We say that $\F_q$ is a splitting field of $g(X)$. 
In general, the a splitting field of a polynomial $f$ over a field $K$ is a the smallest field $L 
\supseteq K$ containing all the roots of $f$. Splitting fields always exists, and they are unique 
up to isomorphisms. This yields the following result.
\begin{result}
	For any given prime $p$ and positive integer $n$ there exists a finite field of size $q = 
	p^n$. Any two such finite field are isomorphic to the splitting field of $X^q - X$ over $\F_p$.
\end{result}
Therefore, we can always talk about \textit{the} finite field of a given size. Since 
$\fieldchar(\F_q) = p$ it is easy to check that $(a + b)^p = a^p + b^p$. This yields the famous 
automorphism 
\[
	\begin{array}{lrll}
		\phi_p: & \F_q & \rightarrow & \F_q \\
		& a & \mapsto & a^p
	\end{array}
\]
called \textit{\textbf{Frobenius automorphism}}. Different powers of the Frobenius are defined by 
composition, e.g. $\phi^2 = \phi \circ \phi$. In fact we have a cyclic group $G = \{ 1, \phi, 
\phi^2, \dots, \phi^{n - 1}\}$ of order $n$. This group is called the Galois group of the 
extension $\F_p \subseteq \F_q$, and is denoted by $\gal(\F_q / \F_p)$. For every element $\sigma 
\in G$ there is a subset $F \subseteq \F_q$ such that $\sigma(a) = a$ for all $a \in F$. One can 
check that $F$ is a field. We call $F$ the fixed field of $\sigma$. Similarly, every subgroup of 
$G$ has a fixed field. In fact, it can be shown that there is a one-to-one correspondence between 
the subgroups of $G$ and subfields of $\F_q$. The subgroups of $G$ are unique and correspond to 
the divisors of $n$. This translate, via the above correspondence, to the following result.
\begin{result}
	For ever divisor $m \mid n$ there is exactly one subfield of $\F_q$ of size $p^m$. Conversely, 
	every subfield of $\F_q$ is of size $p^m$ with $m \mid n$.
\end{result}
Let $K$ be an arbitrary field, and let $G \subseteq K^*$ be a finite subgroup of size $n$. Let $a 
\in G$ be an element with maximal order $m$. Then the order of every other element divides $m$. In 
fact, if $m_1 > 1$ is the order of an element $b \in G$, and $m_1$ is coprime to $m$ then $ab$ has 
order $m_1m > m$ which contradict the assumption of maximality of $m$. Therefore, every element of 
$G$ is a root of $g(X) = X^m - 1$. Since, over a field, $g(X)$ can have at most $m$ roots we have 
$m = n$. So we have found a generator for $G$, hence $G$ is cyclic.
\begin{result}
	The multiplicative group $\F_q^*$ is cyclic.
\end{result}
A generator of the group $\F_q^*$ is called a primitive element of $\F_q$. If $a \in \F_q$ is a 
primitive element then $a^r$ is also a primitive element for all $r$ coprime to $q - 1$. 
Therefore, $\F_q$ has exactly $\varphi(q - 1)$ primitive elements, where $\varphi$ is the Euler's 
totient function. 


\section{Irreducible polynomials}

Let $F \subseteq E$ be an extension of finite fields. We say that the extension is algebraic if 
for every element $a \in E$ there exists a polynomial $g$ over $F$ such that $g(a) = 0$. We define 
the \textit{\textbf{minimal polynomial}} of an element $a \in E$ to be a monic polynomial $g$ over 
$F$ of minimal degree with $g(a) = 0$. The minimality condition on the degree implies that minimal 
polynomials are always irreducible. Indeed, if $g = h_1h_2$ with $\deg h_1, \deg h_2 > 0$ then 
$g(a) = h_1(a)h_2(a) = 0$, and hence say $h_1(a) = 0$ with $\deg h_1 < \deg g$ which is a 
contradiction. 

Another way of introducing minimal polynomials is as follows. As mentioned before, 
every ideal in $F[X]$ can be written as $\ang{f}$ for some $f \in F[X]$. This is, in fact, the 
result of $F[X]$ being an Euclidean domain; i.e. for every $a, b \in F[X]$ with $g \ne 0$, there 
are $q, r \in F[X]$ such that $a = bq + r$ and either $r = 0$ or $\deg r < \deg b$. Let $I \subset 
F[X]$ be an ideal, and let $f \in I$ be a polynomial with lowest degree. We can assume that $f$ is 
monic. Given $g \in F[X]$ we can write $g = fq + r$. If $r \ne 0$ then $r$ then $r = g = fq \in I$ 
and it has a lower degree than $f$ a contradiction. Therefore, $f$ divides every polynomial in 
$I$, and hence $I = \ang{f}$. Now, given $a \in E$ let $I$ be the set of all $g \in F[X]$ such 
that $g(a) = 0$. One readily checks that $I$ is an ideal. Write $I = \ang{f}$, and define $f$ as 
the minimal polynomial of $a$. From this we see that minimal polynomials are unique. It is easy to 
check that the above two definitions are equivalent. The latter yields the following result.
\begin{result}
	Let $F \subseteq E$ be finite field extensions, and let $f \in F[X]$ be the minimal polynomial 
	of an element $a \in E$. Then for any $g \in F[X]$ we have $g(a) = 0$ if and only if $f \mid g$.
\end{result}
One of the interesting polynomials over $\F_q$ is $g(X) = X^{q^r} - X$ for a given $r > 0$. Suppose 
that an irreducible polynomial $f \in \F_q[X]$ of degree $m$ divides this polynomial. The the two 
polynomials have a common root $a$ in the splitting field of $f$ over $\F_q$. Since $a^{q^r} = a$ 
we have the extensions $\F_q \subseteq \F_{q^m} \subseteq \F_q^r$ hence $m \mid r$. Conversely, if 
$m \mid r$ then we have the above extensions and $\F_{q^m}$ is the splitting field of $f$. So $f$ 
and $g$ have a common root $a \in \F_{q^r}$. But $f$ is the minimal polynomial of $a$ over $\F_q$ 
hence $f \mid g$. So we have proved the following. 
\begin{result}
	Let $f$ be an irreducible polynomial of degree $m$ over $\F_q$, and let $g(X) = X^{q^r} - X$. 
	Then $f \mid g$ if and only if $m \mid r$.
\end{result}
The above result say that for a given $r > 0$, $g$ is a the product of all irreducible polynomials 
who's degrees divide $r$. An immediate application of this result is testing for irreducibility. A 
polynomial $f$ of degree $m$ is irreducible if and only if
\begin{enumerate}
	\setlength\itemsep{0mm}
	\item[i]. $f$ divides $X^{q^m} - X$,
	\item[ii]. $\gcd(X^{q^{m / t}} - X, f) = 1$ for all prime divisors $t$ of $m$.
\end{enumerate}
An interesting observation about irreducible polynomials over finite fields is that any extension 
containing one root of an irreducible polynomial contains all the other roots as well. More 
precisely, if $f(X) = X^m + a_{m - 1}X^{m - 1} + \cdots + a_0$ is an irreducible polynomial over 
$\F_q$, and $b \in F_{q^m}$ is a root of $f$ then we have $b^m + a_{m - 1}b^{m - 1} + \cdots + a_0 
= 0$. Raising both sides to the power of $q^i$ for any $1 \le i \le m - 1$ we get $(b^{q^i})^m + 
a_{m - 1}(b^{q^i})^{m - 1} + \cdots + a_0 = 0$. One checks that the distinct elements $b, b^q, 
\dots, b^{q^{m - 1}}$ are all the roots of $f$. These elements are called the conjugates of $b$. 
More generally, given an extension $\F_q \subseteq \F_{q^m}$, and an element $a \in \F_{q^m}$ we 
define the conjugates of $a$ with respect to $\F_q$ as $a, a^q, \dots, a^{q^{m - 1}}$. The 
terminology comes from the action of the elements of $\gal(\F_{q^m} / \F_q) = \{1, \sigma, 
\sigma^2, \dots, \sigma^{m - 1}\}$, where $\sigma^i(x) = x^{q^i}$, on $a$. From the above we also 
see that $\F_{q^m}$ is the splitting field of $f$ over $\F_q$. Therefore, two irreducible 
polynomials of the same degree have isomorphic splitting fields. 

From the beginning we have implicitly assumed that the Galois group $\gal(\F_{q^m} / \F_q)$, which 
is defined to be the group of all automorphisms $\alpha: \F_{q^m} \rightarrow \F_{q^m}$ over 
$\F_q$, consists only of $\sigma^i$ defined above. This is always the case for finite fields. In 
fact, let $\alpha$ be an arbitrary automorphism of $\F_{q^m}$ over $\F_q$. Also let $\beta$ be a 
primitive element of $\F_{q^m}$, and let $f$ be its minimal polynomial of $\F_q$. So $0 = 
\alpha(f(\beta)) = f(\alpha(\beta))$ hence $\alpha(\beta)$ is also a root of $f$. Since all other 
roots of $f$ are conjugates of $\beta$ we must have $\alpha(\beta) = \beta^{q^i}$ for some $0 \le i 
\le m - 1$. Also since $\beta$ is a primitive element we have $\alpha(a) = a^{q^i}$ for all $a \in 
\F_{q^m}$.

\paragraph{Cyclotomic polynomials.} Let $r$ be a positive integer such that $r \mid q - 1$. Then 
there is an element $\zeta \in \F_q$ of order $r$, namely $g^{(q - 1) / r}$ for some generator $g 
\in \F_q^*$. The element $\zeta$ is called a primitive $r$th root of unity. Also for all $1 \le i 
< r$ coprime to $r$, $\zeta^i$ is also a primitive $r$th root of unity. Define the $r$th Cyclotomic 
polynomial as
\[ \Phi_r(X) = \prod_{\substack{1 \le i < r \\ \gcd(i, r) = 1}}(X - \zeta^i). \]
we obviously have $\deg \Phi_r = \phi(r)$ where $\phi$ is the Euler function. The polynomial 
$\Phi_r$ is square-free by definition. Let $f$ be the minimal polynomial of $\zeta$ over $\F_p$, 
and let $d$ be the order of $p$ in the multiplicative group $\Z / r\Z$. Then we know that $d \mid
\phi(r)$ by group theory. As before, $\zeta^{p^i}$ is also a root of $f$ for all $0 \le i < 
\phi(r)$. But only $d$ of these elements are distinct, namely $A = \{ \zeta, \zeta^{p^1}, \dots, 
\zeta^{p^{d - 1}} \}$. So $f$ has degree $d$. We can repeat the same process for the minimal 
polynomial of an element $\zeta^{p^i}$ not in $A$, and append the next set of distinct powers to 
$A$, and so on. All these minimal polynomials divide $\Phi_r$. This yields the following.
\begin{result}
	Let $r$ be a positive integer such that $r \mid q - 1$, and let $d$ be the order of $q$ in $\Z 
	/ r\Z$. Then $\Phi_r$ factors into $\phi(r) / d$ irreducible polynomials of the same degree $d$.
\end{result}
 

\section{Traces and Norms}

Let $F = \F_q$, and $E = \F_{q^m}$ be an extension of $F$. The trace map from $E$ to $F$ is defined 
as
\[
	\begin{array}{lrll}
		\trace_{E/F}: & E & \rightarrow & F \\
		& a & \mapsto & a + a^q + \cdots + a^{q^{m - 1}}.
	\end{array}
\]
So the trace of an element is simply the sum of its conjugates. One hidden fact in the above 
definition is that the image of the trace is actually contained in $F$. This is a direct 
consequence of the fact that trace is fixed by all $\sigma \in \gal(E/F)$. Indeed, 
\begin{align*}
	\trace_{E/F}(a)^q &= (a + a^q + \cdots + a^{q^{m - 1}})^q \\
	& = a^q + \cdots + a^{q^{m - 1}} + a \\
	& = \trace_{E/F}(a). 
\end{align*}
One can easily check that $\trace_{E/F}$ is a linear map over $F$, or an $F$-linear map, 
considering both $E, F$ as vector spaces over $F$; i.e. 
\[ \trace_{E/F}(a\alpha + b\beta) = a\trace_{E/F}(\alpha) + b\trace_{E/F}(\beta) \quad a, b \in F 
\text{ and } \alpha, \beta \in E. \]
This mean $\trace_{E/F}(a) = ma$ for all $a \in F$. An element $a \in E$ is in the kernel $K$ of 
the trace map if it is a root of the polynomial $X + X^q + \cdots + X^{q^{m - 1}}$. But this 
polynomial has at most $q^{m - 1}$ roots. So $\#K \le q^{m - 1}$ hence the image of $\trace_{E/F}$ 
has size larger than $q$. Therefore, the trace map is surjective.

We saw before that every automorphism of $\F_{q^m}$ is of the form $x \mapsto x^{q^i}$ for some $0 
\le i \le m - 1$. This extends to the case of trace maps as follows. Define $\ell_b(a) = 
\trace_{E/F}(ab)$ for $b \in E$ and all $a \in E$. If $a \ne a'$ then $\ell_a - \ell_{a'} = 
\trace_{E/F}(ab) - \trace_{E/F}(ab) = \trace_{E/F}((a - a')b)$ which is not zero for some $b$ as 
the trace in onto. Therefore, $\ell_a \ne \ell_{a'}$. Also there are only a finite number of 
$F$-linear maps $E \rightarrow F$. In fact, every such linear map is determined by assigning 
elements of a given basis of $E$ to elements of $F$. So there are $q^m$ of such maps. But there are 
the same number of maps $\ell_b$ as well. Therefore, every $F$-linear map $E \rightarrow F$ is of 
the form $\ell_b$ for some $b \in E$.

Let $E, F$ be as above. The norm map from $E$ to $F$ is defined 
as
\[
\begin{array}{lrll}
\norm_{E/F}: & E & \rightarrow & F \\
& a & \mapsto & aa^q \cdots a^{q^{m - 1}} = a^{(q^m - 1) / (q - 1)}.
\end{array}
\]
Again the image of $\norm_{E/F}$ is always in $F$. From the definition we have 
\[ \norm_{E/F}(ab) = \norm_{E/F}(a)\norm_{E/F}(b) \quad a,b \in E. \]
This means that norm is a homomorphism $E^* \rightarrow F^*$ of groups. We also have 
$\norm_{E/F}(a) = a^m$ for all $a \in F$. Like trace, norm is also onto: the kernel of 
$\norm_{E/F}$ is a subset of the roots of the polynomial $X^{(q^m - 1) / (q - 1)}$. So the kernel 
has size smaller than $(q^m - 1) / (q - 1)$ hence the image of the map has size larger 
than $q - 1$. The norm and trace maps are both transitive in the following sense.

\begin{result}
	For a chain of extensions $F \subset K \subset E$ of finite fields
	\[ \trace_{E/F}(a) = \trace_{K/F}(\trace_{E/K}(a)), \quad \norm_{E/F}(a) = 
	\norm_{E/K}(\norm_{K/F}(a)) \]
	for all $a \in E$.
\end{result}

\section{Algebraic closures}

In this section, we discuss the basic concepts of algebraic closures and their construction. We 
will also discuss our computational approach to dealing with algebraic closures of finite fields.

A field $L$ is said to be algebraically closed if every non-constant polynomial in $L[X]$ has a 
root in $L$. This is equivalent to saying that every non-constant polynomial splits into linear 
factors over $L$. An \textit{\textbf{algebraic closure}} of a field $K$, denoted by $\overline{K}$, 
is an algebraic extension of $K$ that is algebraically closed.

Given a field $K$ one can build an algebraically closed filed containing $K$ as 
follows. We first build a field $K_1$ such that every polynomial in $K[X]$ has a root in $K_1$. Let 
$\mathcal{S}$ be the set of all polynomials $f \in K[X]$, and let $\mathcal{X}$ be the set of 
variables $\{X_f\}_{f \in \mathcal{S}}$. So we have introduced a variable for each polynomial. Now 
form the ring $K[\mathcal{X}]$, and let $I \subset K[\mathcal{X}]$ be the ideal generated by all 
the polynomials $f(X_f)$. We claim that $I$ is not the unit ideal. If it is, then there is a finite 
linear combination
\[ g_1f_1(X_{f_1}) + \cdots + g_nf_n(X_{f_n}) = 1, \qquad g_i \in K[\mathcal{X}]. \]
This equation involves only a finite number of variables, say $X_{f_1}, \dots, X_{f_N}$. Therefore, 
rewriting the equation gives
\[ \sum_{i = 1}^n g_i(X_{f_1}, \dots, X_{f_N})f_i(X_{f_i}) = 1. \]  
There exists a finite extension $E$ of $K$ in which each $f_i$ has a root. Let $a_i \in E$ be a 
root of $f_i$. Then substituting $a_i$ for $X_{f_i}$ in the above equation will give $0 = 1$ in 
$E$, which is a contradiction. So a simple application of Zorn's lemma gives $I \subseteq 
\mathfrak{m}$ for some maximal ideal $\mathfrak{m}$ of $K[\mathcal{X}]$. So $K_1 = K[\mathcal{X}] / 
\mathfrak{m}$ is a field containing $K$. Also every polynomial $f \in K[X]$ has a root in $K_1$ by 
construction. Repeating the same process for $K_1$, and so on, we obtain a tower
\[ K = K_0 \subseteq K_1 \subseteq K_2 \subseteq \cdots \]
in which every non-constant polynomial in $K_n[X]$ has a root in $K_{n + 1}$ for all $n \ge 0$. Now 
define $K_\infty$ to be the union of all these extensions. One can easily check that $K_\infty$ is 
a field. Let $f \in K_\infty[X]$. Then the coefficients of $f$ are in $K_n$ for a large enough $n$. 
So $f$ has a root in $K_{n + 1} \subseteq K_\infty$, hence $K_\infty$ is algebraically closed.

So given a field $K$ there is an algebraically closed extension $K \subseteq E$. Define $F 
\subseteq E$ as the union of all subfields of $E$ that are algebraic over $K$. It is easy to check 
that $F$ is algebraic over $K$ and it is algebraically closed. Therefore, $F$ is an algebraic 
closure of $K$. It can be shown that algebraic closures are unique up to isomorphism of fields.

\paragraph{Finite fields.} It is a simpler and more intuitive situation for algebraic closures over 
finite fields. Starting from $\F_p$, we know that every irreducible polynomial $f \in \F_p[X]$ of 
degree $n$ has a root in $\F_{p^n}$. So adapting the above general construction of adding roots of 
polynomials we see that we only need to consider extensions $\F_{p^2}, \F_{p^3}, \F_{p^4}, \dots$. 
From previous sections we know that $\F_{p^m} \subseteq \F_{p^n}$ if and only if $m \mid n$. So,  
there are inclusions $\F_{p^{n_1}} \subseteq \F_{p^{n_1n_2}}$ and $\F_{p^{n_2}} \subseteq 
\F_{p^{n_1n_2}}$ for every $n_1, n_2 > 0$. So the above set of finite fields is partially ordered 
by inclusion, and we can talk about the union of any two finite fields. Then we have 
\[ \overline{\F_p} = \bigcup_{i \ge 1} \F_{p^i}. \]
In fact, let $K = \cup_{i \ge 1} \F_{p^i}$, and let $a \in K$. Then $a \in \F_{p^n}$ for some $n$, 
hence $a$ is algebraic over $\F_p$. Also every irreducible polynomial $f \in K[X]$ has coefficients 
in $\F_{p^n}$ for a large enough $n$. Then $f$ has a root in $\F_{p^n}[X] / \ang{f} \cong 
\F_{p^{mn}} \subseteq K$ where $m = \deg f$.


\bibliographystyle{plain}
\bibliography{references}