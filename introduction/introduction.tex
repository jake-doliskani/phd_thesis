\graphicspath{{introduction/}}

\chapter{Introduction}

The theory of finite fields has found much attention during recent decades, most notably due to 
developments in the branches of computer science such as cryptography, coding theory, switching 
circuits, and combinatorics \cite{lidl2012applied}. In applications, one almost always encounters 
finite fields that are extensions of the so-called prime fields. For example, in public-key 
cryptography, there are cryptosystems that are based on the group of points on an algebraic curve. 
To have a secure system one must go through the process of choosing a random secure curve. A good 
class of candidates for such curves are hyperelliptic curves. One of the efficient methods of 
choosing a secure hyperelliptic curve requires counting the number of points on the curve, which in 
turn requires building a large number of successive extensions over a small prime field 
\cite{df10, GaSc12}. 

The natural need for extensions usually arises from the need for manipulating polynomials and their 
roots. Once the extension are built, the next step is to be able to move elements from one 
extension to the other. This is where algebraic closures come to the fore. From a computational 
point of view, an algebraic closure of a given field can be seen as a large enough finite extension 
that contains the roots of a given finite set of polynomials. Computation in algebraic closures 
has been considered by others, e.g. \cite{Steel2010342} and references therein. In this chapter, we 
briefly review some basic properties of finite fields, and discuss some notations on the running 
time complexities of some basic arithmetics over them. At the end, we give an overview of our 
computational approach to algebraic closures.


\section{Notations}

In this section, we give a very brief review of finite fields concepts and notations used in the 
subsequent chapters. For a more detailed preliminary on finites fields we refer the reader to 
Appendix \ref{appendix:finite_fields}. Given a prime number $p$, we denote the prime finite field 
with $p$ elements by $\F_p$. The prime number $p$ is called the characteristic of the field. 

\paragraph{Irreducible polynomials.}
The univariate polynomial ring over $\F_p$ is denoted by $\F_p[X]$. The elements of $\F_p[x]$ are 
of the form $f(X) = a_{n}X^n + a^{n - 1}X^{n - 1} + \cdots + a_0$ where $a_i \in \F_p$. The degree 
of $f$, denoted by $\deg f$, is $n$. We call $f$ monic if its leading coefficient $a_n$ is 1. The 
division and remainder operation in $\F_p[X]$ is done using the usual polynomial arithmetic. A 
polynomial $f \in \F_p[X]$ is reducible if it can be written as $f = gh$ for polynomials $g, h \in 
\F_p[X]$ of degree $> 0$; otherwise it is called \textit{irreducible}. For a monic irreducible 
polynomials $f$ of degree $n$ the quotient $\F_p[X] / \ang{f}$ is a finite field extension of 
$\F_p$ of degree $n$. We usually denote such an extension by $\F_q$ where $q = p^n$.  

\paragraph{Operation complexities}
The nature of the algorithms presented in this document suggests an algebraic complexity model. 
This means we will count operations $\{+, -, \times, \div \}$ in a based field such as $\F_p$ or 
$\F_q$. Consequently, we have to choose a representation for elements of field extensions. For 
example, one could proceed implementing algorithms using normal basis representation, in which 
case the complexity estimates are stated differently. However, we shall use the monomial basis 
representation, since in particular our implementations and result comparisons are based on 
systems such as NTL \cite{shoup2003ntl}, Magma \cite{MAGMA}, and FLINT \cite{hart2010flint}. In 
the monomial representation, the finite field or even ring extension are represented by quotients. 
For example, the finite field $\F_{q^n}$ is represented by $\F_q[X] / \ang{f}$ where $f$ is a 
monic irreducible polynomial of degree $n$ in $\F_q[X]$. So an element $a \in \F_{q^n}$ is a 
polynomial of degree less than $n$ in $\F_q[X]$. 

One of the most fundamental operations in all algebraic systems is \textit{polynomial 
multiplication}. We denote the upper bound for the cost of multiplying two polynomials of degree 
$n$ by $\MM(n)$. This means for a given field $K$ and two polynomials $f, g \in K[X]$ of degree 
$n$, $fg$ can be computed using $\MM(n)$ multiplications in $K$. The best known upper bound for 
$\MM(n)$ is $O(n\log(n)\log\log(n))$ \cite{Schonhage1971,CaKa91}. This is done using Fast Fourier 
Transform. The idea is to reduce the polynomial multiplication to point-wise vector-vector 
multiplication using an invertible change of representation. See \cite[Chapter~8]{GaGe03} for more 
details. We assume that $\MM(n)$ is a super-linear function, i.e. not bounded by any linear 
function.

Another fundamental operation is \textit{matrix multiplication}. Given two square matrices of size 
$n$ over a field $K$ we assume they can be multiplied using $O(n^\omega)$ multiplications in $K$, 
where $\omega$ is called the linear algebra exponent. A naive implementation would give $\omega = 
3$. A slightly better algorithm due to Strassen \cite{Strassen1969} gives a bound $\omega \le 2.81$.
The algorithm uses a divide and conquer approach to reduce the multiplication problem to sub-blocks 
of the given matrices. The smallest known bound is $\omega \le 2.37$ due to Coppersmith and 
Winograd \cite{CoWi90}. 

Given polynomials $f, g, h \in K[X]$ of degree $n$, the \textit{modular composition} problem is to 
compute $f(g) \mod{h}$. We denote the upper bound for modular composition by $\CC(n)$. In a boolean 
complexity model, in which bit or word operation are counted, there is an algorithm due to 
Kedlaya and Umans \cite{KeUm11, Umans08} that runs in $O(n^{1+\varepsilon}\log(p)^{1+o(1)})$ 
operations. Since we do not follow a boolean model, and there is still no competitive  
implementation of their algorithm that we know of, we opt for the well-known algorithm of Brent and 
Kung \cite{BrKu78}. Their algorithm gives the bound $O(n^{(\omega+1)/2})$. We can take $\omega \le 
2.37$ using the Coppersmith and Winograd algorithm which gives $\CC(n)=O(n^{1.69})$. There is the 
slightly better bound $\CC(n)=O(n^{1.67})$ due to Pan \cite{HuPa98}, which uses rectangular matrix 
multiplication.

\paragraph{Algebraic closures.}
A field extension $K \subseteq L$ is said to be algebraic if every element $a \in L$ is a root of a 
monic polynomial in $K[X]$. A field $K$ is algebraically closed if $L = K$ for every algebraic 
extension of $L$. An algebraic closure of a field $K$, denoted by $\overline{K}$, is an 
algebraically closed algebraic extension of $K$. 

Let $E, F$ be extensions of $K$. Assume that $E, F$ are contained in some larger field $L$. We 
define the \textit{compositum} of $E, F$, denoted by $EF$, to be the smallest subfield of $L$ 
containing both $E, F$. Generally, we can define the compositum of a family $\{ F_i \}_{i \in I}$ 
of subfields of $L$ to be the smallest subfield of $L$ containing all $F_i$. The compositum of $E, 
F$ is not defined unless we have embeddings of both fields into a common field $L$. Let $m = [E: 
K]$ and $n = [F: K]$. We have the following diagram of extensions:
\begin{center}
	\begin{tikzpicture}[node distance = 2cm]
		\node (L) {$L$};
		\node[below = 1cm of L] (EF) {$EF$};
		\node[below left of = EF] (E) {$E$};
		\node[below right of = EF] (F) {$F$};
		\node[below right of = E] (K) {$K$};
		\draw (L) edge (EF);
		\draw (EF) edge node[above left] {$\le n$} (E) edge node[above right] {$\le m$} (F);
		\draw (K) edge node[below left] {$m$} (E) edge node[below right] {$n$} (F);
	\end{tikzpicture}
\end{center}
From the diagram $[EF: K] \le mn$ and $m \mid [EF: K]$ and $n \mid [EF: K]$. So if $\gcd(m, n) = 1$ 
then $[EF: K] = mn$. Therefore, if $E, F$ are finite fields $\F_{p^n}, \F_{p^m}$ with $\gcd(m, n) = 
1$ then the compositum $EF$ is the finite field $\F_{p^{mn}}$. This allows us to define the 
algebraic closure of $\F_p$ as the union
\[ \overline{\F_p} = \bigcup_{i \ge 1} \F_{p^i}. \]


\section{Our approach}

Let $\F_q$ be a finite field with $q = p^n$ a prime power. We divide the construction of 
$\overline{\F_q}$ into two major steps: building towers, and gluing towers. Let us explain what 
these mean. Let $\ell$ be a prime number, and let 
\[ \F_q \subset \F_{q^\ell} \subset \F_{q^{\ell^2}} \subset \cdots \]
a sequence of extensions each of degree $\ell$. We call this an \emph{$\ell$-adic tower}. Define 
the \emph{$\ell$-adic closure} of $\F_q$ to be the union
\[ \F_q^{(\ell)} = \bigcup_{i\ge 0}\F_{q^{\ell^i}}. \]
Assume we have build these towers for different prime $\ell$. Then we can glue them together to get 
the algebraic closure. By gluing here we mean computing composita. Figure \ref{figure:alg-closure}  
shows the above structure of $\overline{\F_q}$. The solid circles in the middle are the composita.
\begin{figure}
	\begin{center}
		\tikzset{
			dotstyle/.style={inner sep = 2pt, outer sep = 2pt, circle, fill = lightgray},
			edgetower/.style={thick},
			edgecomp/.style={thick, lightgray}
		}
		\begin{tikzpicture} %, every node/.style={inner sep = 2pt, scale=0.8}]
			\coordinate (T2) at (-2, 0.5);
			\node (Fq) at (0, 0) {$\F_q$};
			\node (Fq2) at ($(Fq) + (T2)$) {$\F_{q^2}$};
			\node (Fq4) at ($(Fq2) + (T2)$) {$\F_{q^4}$};
			\node (Fq2l) at ($(Fq4) + (T2)$) {$\F_q^{(2)}$};
			%---------------------
			\coordinate (T3) at (-0.7, 2);
			\node (Fq3) at ($(Fq) + (T3)$) {$\F_{q^3}$};
			\node (Fq9) at ($(Fq3) + (T3)$) {$\F_{q^9}$};
			\node (Fq3l) at ($(Fq9) + (T3)$) {$\F_q^{(3)}$};
			%---------------------
			\coordinate (T5) at (0.7, 2);
			\node (Fq5) at ($(Fq) + (T5)$) {$\F_{q^5}$};
			\node (Fq25) at ($(Fq5) + (T5)$) {$\F_{q^{25}}$};
			\node (Fq5l) at ($(Fq25) + (T5)$) {$\F_q^{(5)}$};
			%---------------------
			\coordinate (Tl) at (2, 1);
			\node[blue] (Fql) at ($(Fq) + (Tl)$) {$\F_{q^\ell}$};
			\node[blue] (Fql2) at ($(Fql) + (Tl)$) {$\F_{q^{\ell^2}}$};
			\node[blue] (Fqll) at ($(Fql2) + (Tl)$) {$\F_q^{(\ell)}$};
			%---------------------
			\node[dotstyle] (dot1) at ($(Fq2) + (Fq3) - (Fq)$){};
			\node[dotstyle] (dot2) at ($(Fq4) + (dot1) - (Fq2)$){};
			\node[dotstyle] (dot3) at ($(Fq2) + (Fq5) - (Fq)$){};
			\node[dotstyle] (dot4) at ($(Fq3) + (Fq5) - (Fq)$){};
			\node[dotstyle] (dot5) at ($(Fq3) + (Fql) - (Fq)$){};
			\node[dotstyle] (dot6) at ($(Fq5) + (Fql) - (Fq)$){};
			%---------------------
			\draw 
			(Fq) 
			edge[edgetower] (Fq2)
			edge[edgetower] (Fq3)
			edge[edgetower] (Fq5)
			edge[edgetower, blue] (Fql)
			(Fq2)
			edge[edgetower] (Fq4)
			edge[edgecomp] (dot1)
			(Fq4)
			edge[edgetower, dotted] (Fq2l)
			edge[edgecomp] (dot2)
			(dot1)
			edge[edgecomp] (dot2)
			(Fq3)
			edge[edgetower] (Fq9)
			edge[edgecomp] (dot1)
			edge[edgecomp] (dot4)
			(Fq9)
			edge[edgetower, dotted] (Fq3l)
			(Fq5)
			edge[edgetower] (Fq25)
			edge[edgecomp] (dot4)
			edge[edgecomp] (dot6)
			(Fq25)
			edge[edgetower, dotted] (Fq5l)
			(Fql)
			edge[edgetower, blue] (Fql2)
			edge[edgecomp] (dot6)
			(Fql2)
			edge[edgetower, blue, dotted] (Fqll)
			(dot3)
			edge[edgecomp] (Fq2)
			edge[edgecomp] (Fq5)
			(dot5)
			edge[edgecomp] (Fq3)
			edge[edgecomp] (Fql);
		\end{tikzpicture}
	\end{center}
	\caption{Algebraic closure of $\F_q$}
	\label{figure:alg-closure}
\end{figure}

\paragraph{$\ell$-adic towers.} Let $F_i = \F_{q^{\ell^i}}$ be the level $i$ of the $\ell$-adic 
tower for an integer $i > 0$. The goal is to be able to compute in $F_i$ in quasi-linear time in 
the extension degree $\ell^i$. For this we need first to construct the tower, and then be able to 
move up and down to different levels of the tower efficiently. 

In many cases, building extensions and isomorphisms of finite fields requires taking roots of 
elements. In Chapter \ref{chapter:high-ext-roots} we discuss taking arbitrary roots over extensions 
of finite fields. The idea is to reduce the problem to root extraction in subfields using 
trace-like computations. For example, using our algorithm, the complexity of taking square roots in 
$\F_q$ is an expected $O(\MM(n)\log(p) + \CC(n)\log(n))$ operations in $\F_p$. Computing roots in 
finite fields was previously considered by others like Cipolla \cite{Williams72} and Tonelli-Shanks 
\cite{Shanks1972}. Kaltofen and Shoup \cite{KaltofenShoup1997} give an Equal Degree Factorization 
(EDF) algorithm that can be used for computing roots.

The case $\ell = 2$ of the $\ell$-adic towers, which is of particular interest, is treated 
separately in Chapter \ref{chapter:2ext-op}. Our algorithms presented there for construction of the 
tower and performing basic operations inside the tower are all quasi-linear time. The following 
table summarizes the complexities of our algorithms for some well-known operations for this case.

\begin{center}
	\begin{tabular}{c|c}
		Operation & Cost \\
		\hline
		addition / subtraction & $O(n)$\\
		multiplication & $\MM(n)+O(n)$\\
		inversion & $O(\MM(n))$\\
		quadratic residuosity & $O(\MM(n)+\log(q))$ \\
		square root &  $O(\MM(n)\log(nq))$\ \ (expected) \\
		isomorphism &  $O(n + \log(n)\log(q))$\ \ (expected)
	\end{tabular}
\end{center}
Note the improvement of the square root complexity here compared to the one for general finite 
fields which has a $\CC(n)$ term. Another special case of the $\ell$-adic towers is $\ell = p$, 
called the Artin-Schreier tower. The algorithms for this case were presented in \cite{DeSc12}.

Chapter \ref{chapter:ell-adic} discusses the towers for a general $\ell$. The field 
$\F_{q^{\ell^i}}$ can be represented by two natural bases over $\F_q$. For a univariate 
representation of the form $\F_q[X_i]/\ang{Q_i}$, where $Q_i \in \F_q[X_i]$ is a monic irreducible 
polynomial of degree $\ell^i$, we get the monomial basis
\[ \uu_i = (1,x_{i},x_{i}^2,\ldots,x_{i}^{\ell^{i}-1}) \] 
for $\F_{q^{\ell^i}}$ over $\F_q$. Here $x_i$ is the residue class of $X_i$ modulo $Q_i$. But we 
can also consider $\F_{q^{\ell^i}}$ as a degree $\ell$ extension over $\F_{q^{\ell^{i - 1}}}$, 
i.e. as a bivariate quotient
\[ \F_q[X_{i-1},X_i]/\ang{Q_{i-1}(X_{i-1}), T_i(X_{i-1},X_i)} \]
where $T_i$ is a monic polynomial of degree $\ell$ in $X_i$, and of degree less than $\ell^{i-1}$ 
in $X_{i-1}$. This gives the basis
\[ \bb_{i} = (1,\ldots,x_{i-1}^{\ell^{i-1}-1},\ldots,x_i^{\ell-1},\ldots, x_{i-1}^{\ell^{i-1}-1}
x_i^{\ell-1}). \]
for $\F_{q^{\ell^i}}$ over $\F_q$. The complexity of constructing the tower, which requires some 
initialization and finding $Q_i, T_i$, is close to quasi-linear. Using the usual algorithms, basic 
operations like multiplication and inversion in $\F_q[X_i]/\ang{Q_i}$ are done using respectively 
$O(\MM(\ell^i))$ and $O(\MM(\ell^i)\log(\ell^i))$ operations in $\F_q$. These are quasi-linear in 
the extension degree $\ell^i$. Moving up and down in the tower is done using repeated applications 
of two operations called \textit{lift} and \textit{push}. Lifting refers to the change of basis 
from $\bb_i$ to $\uu_i$ while pushing is the inverse transformation. These operations are also 
very close to being quasi-linear. The following table summarizes our main complexity results.

\vspace*{4mm}
\resizebox{\textwidth}{!}{$
	\def\arraystretch{1.2}
	\begin{array}{c|cccc}
		\text{Condition} & \text{Initialization} & Q_i, T_i & \text{Lift, push}\\
		\hline \hline
		q = 1 \bmod \ell & O_e(\log(q))  & O(\ell^i) & O(\ell^i) \\
		q = -1 \bmod \ell & O_e(\log(q)) & O(\ell^i) & O(\MM(\ell^i) \log(\ell^i)) \\
		- & O_e (\ell^2+\MM(\ell) \log(q)) & O(\MM(\ell^{i+1})\MM(\ell)\log(\ell^i)^2) & O( 
		\MM(\ell^{i+1})\MM(\ell)\log(\ell^i))\\
		4\ell \le q^{\sfrac{1}{4}} & O\tilde{_e}(\ell\log^5(q)+\ell^3) \text{~(bit)}  & 
		O_e(\ell^2+\MM(\ell)\log(\ell
		q)+\MM(\ell^i)\log(\ell^i)) & O(\MM(\ell^i) \log(\ell^i)) \\
		4\ell \le q^{\sfrac{1}{4}} & O\tilde{_e}(\ell\log^5(q))  \text{~(bit)}  + 
		O_e(\MM(\ell)\sqrt{q}\log(q)) & O_e(\log(q) + \MM(\ell^i) \log(\ell^i)) & O(\MM(\ell^i) 
		\log(\ell^i))
	\end{array}
$}

\vspace*{4mm}
The probabilistic complexities with expected running time are denoted by $O_e(\ )$. Also
$O\tilde{_e}(\ )$ indicates the additional omission of logarithmic factors. Although in some cases 
there are extra factors of $\ell$, we have achieved quasi-linear time in the degree of the 
extension $\ell^i$ in most of the cases. Our algorithm for constructing towers are inspired by the
previous works of Shoup \cite{Shoup90,shoup94} and Lenstra / De Smit 
\cite{lenstra+desmit08-stdmodels}, and Couveignes / Lercier \cite{couveignes+lercier11}. See the 
introduction of Chapter \ref{chapter:ell-adic} for more details.

\paragraph{Composita.} In Chapter \ref{chapter:alg-closure}, we discuss the composita of fields. 
Let $\F_{p^m} = \F_p[x] / \ang{Q_m(x)}$ and $\F_{p^n} = \F_p[y] / \ang{Q_n(y)}$ be two finite 
fields where $Q_m(x) \in \F_p[x]$ and $Q_n(y) \in \F_p[y]$ are irreducible of comprime degrees $m, 
n > 1$ respectively. Define the \textit{composed product} of $Q_m, Q_n$ as
\[ Q_{mn}(z) = \prod_{\substack{1 \le i \le m \\ 1 \le j \le n}} (z- a_i b_j) \]
where $(a_i)_{1 \le i \le m}$ and $(b_j)_{1 \le j \le n}$ are roots of $Q_m$ and $Q_n$ in the 
algebraic closure of $\F_p$. The polynomial $Q_{mn}$ is irreducible of degree $mn$ in $\F_p[z]$, 
see \cite{Brawley1987}. The field $\F_p[z] / \ang{Q_{mn}(z)}$ is the compositum of $\F_{p^m}$ and 
$\F_{p^n}$, and there exists embeddings $\varphi_x$, $\varphi_y$, and an isomorphism $\Phi$ of the 
form
$$
\begin{array}{rrll}
\varphi_x: & \F_p[x]/\ang{Q_m} & \to & \F_p[z]/\ang{Q_{mn}},\\
\varphi_y: & \F_p[y]/\ang{Q_n} & \to & \F_p[z]/\ang{Q_{mn}},\\
\Phi:&  A=\F_p[x,y]/\ang{Q_m,Q_n} & \to & \F_p[z]/\ang{Q_{mn}} \\
&  xy & \mapsfrom & z.
\end{array}$$
The goal is to compute $\varphi_x$, $\varphi_y$, and $\Phi$ and $\Phi^{-1}$ efficiently. We present 
algorithms for computing $\varphi_x$ and $\varphi_y$ which run using $O(n\MM(m)+m\MM(n))$ 
operations in $\F_p$. Assuming that $m \le n$, $\Phi$ and $\Phi^{-1}$ can be computed using either 
$O(m^2 \MM(n))$ or $O(\MM(mn)n^{1/2}+\MM(m)n^{(\omega+1)/2} )$ operations in $\F_p$ where $\omega$ 
is the linear algebra exponent. Therefore, computing embeddings $\varphi_x, \varphi_y$ is 
quasi-linear in $mn$ while computing the isomorphisms $\Phi, \Phi^{-1}$ has an extra factor of $m$ 
or $n$. Previous algorithms such as the ones presented in \cite{bosma+cannon+steel97}, and 
\cite{LenstraJr91,Allombert02} also compute embeddings of finite fields. They rely on linear 
algebra which results in complexities at least quadratic in the degree of the extensions.

\bibliographystyle{plain}
\bibliography{references}